\documentclass[11pt]{article}

\usepackage{sectsty}
\usepackage{graphicx}
\usepackage{amsmath}
\usepackage{biblatex}
\addbibresource{refs.bib}

% Margins
\topmargin=-0.45in
\evensidemargin=0in
\oddsidemargin=0in
\textwidth=6.5in
\textheight=9.0in
\headsep=0.25in

\title{Shape comparison metrics - a short report}
\author{Martin Metodiev}
\date{\today}

\begin{document}
\maketitle	


% Optional TOC
% \tableofcontents
% \pagebreak

%--Paper--

\section{Introduction}

Here we review some useful metrics. We will focus on shape analysis metrics that incorporate information from both the vertices and connectivity of a mesh dataset \cite{shapedna}, \cite{hyp_w}. Ideally, a shape metric will be independent of the relative spatial location of two objects, and also rotational and scaling variations. An intuitive approach is to use the Hausdorff distance as a measure of similarity between two sets \( A \) and \( B \) is defined as:

\[
d_H(A, B) = \max \left( \sup_{a \in A} \inf_{b \in B} \| a - b \|, \sup_{b \in B} \inf_{a \in A} \| a - b \| \right)
\]

Where:
\begin{itemize}
    \item \( A \) and \( B \) are the two sets (meshes) represented as sets of points,
    \item \( a \) and \( b \) are points in meshes \( A \) and \( B \), respectively,
    \item \( \| a - b \| \) is the Euclidean distance between points \( a \) and \( b \),
    \item The first term \( \sup_{a \in A} \inf_{b \in B} \| a - b \| \) measures the greatest distance from any point in mesh \( A \) to the closest point in mesh \( B \),
    \item The second term \( \sup_{b \in B} \inf_{a \in A} \| a - b \| \) measures the greatest distance from any point in mesh \( B \) to the closest point in mesh \( A \).
\end{itemize}


\pagebreak
\section{Section 1}





\end{document}
